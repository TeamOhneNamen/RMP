% !TEX root = rmp_lernzettel.tex

\section{Lösung für Test 4:}

\subsection{}
\begin{lstlisting}
b	StrCmp
\end{lstlisting}
\paragraph{} Warum nicht bl? Braucht man das lr nicht?
\paragraph*{Nachschlagen:}

\subsection{}
\begin{lstlisting}
pa	= &a;
int *pa = &a;
\end{lstlisting}
\paragraph*{Nachschlagen:}
Kapitel 13.13

\subsection{}
\begin{lstlisting}
pVek = Vek;
int *pVek = Vek[0];
\end{lstlisting}
\paragraph*{Nachschlagen:}
Kapitel 13.13\\
Kapitel 13.18

\subsection{}
\begin{lstlisting}
int var = Vek[3];
int var = *pVek[3];
\end{lstlisting}
\paragraph*{Nachschlagen:}
Kapitel 13.13

\subsection{}
\begin{lstlisting}
printf("%d %d %d", x, &x, px);

-> schreibe 3 Dezimalzahlen:
	x 		-> Wert von x
	&x		-> Adresse von x
	px		-> Adresse auf die px zeigt -> Adresse von x
	
10, 2400, 2400

printf("%d %d %d", &px, *px, *ppx);

-> schreibe 3 Dezimalzahlen:
	&px 	-> Adresse von px
	*px		-> Wert dessen worauf px zeigt -> Wert von x
	*ppx	-> Wert dessen worauf ppx zeigt -> Wert von px -> Adresse von x
	
2000, 10, 2400
\end{lstlisting}
\paragraph*{Nachschlagen:}
Kapitel 13.13

\subsection{}
\begin{lstlisting}
int* api[12];

Array mit der Laenge 12 von Pointern, die auf Integer zeigen 
\end{lstlisting}
\paragraph*{Nachschlagen:}
Kapitel 13.20

\subsection{}
\begin{lstlisting}
Laenge1 = strlen(sP1);

Laenge2	= strlen(sP2[2]);

Laenge3	= strlen(&*(sP2 + 2);
\end{lstlisting}
\paragraph*{Nachschlagen:}