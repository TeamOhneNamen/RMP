% !TEX root = rmp_lernzettel.tex

\section{Lösung für Test 3:}
Gegeben ist folgendes Programmfragment (für die Aufgaben 1-3):
\begin{lstlisting}
		AREA MyData, DATA, align = 8
VarA			DCD		0x17, 17, 0xA, 'A'
VarB			DCB		33, -1, 0x10, 0x20, 0x30, 'a', '1'
\end{lstlisting}
\subsection{}
\begin{lstlisting}
ldr 	r0,= VarA	; Lade den Anfang von VarA in r0
ldr 	r1, [r0]	; Lade das erste Elemente von VarA in r1
ldr 	r2, [r0 #4]	; Lade das zweite Element von VarA in r2 (pre increment)
adds	r1, r1, r2	; Addiere r1 und r2 und speichere das Ergebnis in r1
\end{lstlisting}

\subsection{}
\begin{lstlisting}
VarB = 0x2010
VarA = 0x2000
VarB - VarA = 0x2010 - 0x2000 = 0x0010	; Anfangsadressen werden subtrahiert
\end{lstlisting}

\subsection{}
  \begin{tabular}{ | c | c | c | c | c | c |}
    0x2000 & 0x2001  & 0x2002 & 0x2003 & 0x2004 & 0x2005 \\
    0x17 & 00  & 00 & 00 & 0x11 & 00 \\
  \end{tabular}\\
Zu Beachten: align = 8 $\rightarrow$ Es wird auf 8 Byte aufgefüllt.

\subsection{}
\begin{lstlisting}
bl MyProg
\end{lstlisting}
bl (branch link) springt bedingungslos.

\subsection{}
Speichert die Rücksprungsstelle.
\begin{lstlisting}
r14
\end{lstlisting}.
\paragraph*{Nachschlagen:}
Kapitel 10.1.3.2 

\subsection{}
Der Stackpointer zeigt uns an welcher Stelle wir uns im Stack befinden.
\begin{lstlisting}
r13
\end{lstlisting}

\subsection{}
push{} fügt etwas dem Stack hinzu.
sp (Stackpointer) wird ums eins verringert.
\paragraph*{Nachschlagen:}
Kapitel 9.2

\subsection{}
\begin{lstlisting}
push 	{fp, lr}
mov 	sp, fp
push 	{r1 - r4}	; 8-align -> gerade Anzahl Registern
pop 	{r1 - r4}	; 8-align -> gerade Anzahl Registern
pop		{fp, lr}
bx		lr
\end{lstlisting}
\paragraph*{Nachschlagen:}
Kapitel 9.3.1

\subsection{}
\begin{lstlisting}
push 	{r1 - r4}
bl
...
add		sp, #16
\end{lstlisting}

\subsection{}
Vereinfacht Zugriff auf lokale Daten.
\begin{lstlisting}
r11
\end{lstlisting}
\paragraph*{Nachschlagen:}
Kapitel 10.2.3.1
