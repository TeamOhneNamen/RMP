% !TEX root = rmp_lernzettel.tex


\section{Lösung für Test 1:}
\subsection{}
Wie lautet die Hexadezimalzahl zur Binärzahl?
\paragraph{Lösung}
\begin{lstlisting}
1001 1101 1010
\end{lstlisting}
\begin{lstlisting}
   9    D   10 
\end{lstlisting}
\subsection{}
Wie lautet die Binärzahl zur Dezimalzahl 97
\paragraph{Lösung}
0110 0001
\paragraph{Erklärung}
1*64 + 1*32 + 1*1
\subsection{} 
Addieren Die nebenstehende 8-Bit Binärzahlen:\\
\\
Ist das Carry-Flag gesetzt:\\
Ist das Overflow-Flag gesetzt:
\begin{lstlisting}
 0111 1111
+1011 0011 
\end{lstlisting}
\paragraph{Lösung}
\begin{lstlisting}
   0111 1111
  +1011 0011
------------
U 11111 1110
------------
   0011 0010
\end{lstlisting}
Carry-Flag: 	ja\\
Overflow-Flag:	nein\\

\paragraph{Erklärung}
Das Carry-Flag ist gesetzt dar der hinterste Übertrag auf 1 gesetzt ist.\\
Das Overflow-Flag ist nicht gesetzt dar der hinterste und der vorhinterste Übertrag in Xor-Verbindung 0 ergibt.\\

\subsection{}

Geben Sie den Dezimalwert zur Hexadezimalzahl 7D an.

\paragraph{Lösung}
125

\paragraph{Erklärung}
\begin{lstlisting}
Hex $\rightarrow$          7 |       D\\
Bin $\rightarrow$ 0  1  1  1 | 1 1 0 1\\
Dez $\rightarrow$ / 64 32 16 | 8 4 / 1 = 125\\
\end{lstlisting}

\subsection{}

Wie lautet die 8-Bit-Zweierkompliment-Darstellung zur Dezimal -97?

\paragraph{Lösung}
\begin{lstlisting}
                 -97 = 1001 1111
Zweierkompliment: 97 = 0110 0001
\end{lstlisting}


\paragraph{Erklärung}
-128 + 16 + 8 + 4 + 2 + 1 = -97

$\rightarrow$ Zweierkompliment = Binär invertieren + 0000 0001

\subsection{}

Woran erkennt man bei der Subtraktion zweier Vorzeichenbehafteter Zahlen, ob das berechnete Ergebnis falsch ist?

\paragraph{Lösung}
Overflow-Flag:\\
=1 $\rightarrow$ flasch\\
=0 $\rightarrow$ richtig

\paragraph{Erklärung}
s.o.
\subsection{}
Das Datenfeld...


\subsection{}

Ab Adresse 0x1004 steht folgendes im Speciher (hex.) little endian:
\begin{lstlisting}
14 21 32 A3 A7 F3 FA
\end{lstlisting}
Was steht in r1 nach folgender Sequenz?

\begin{lstlisting}
mov	r0, #0x1006
ldrh	r1, [r0]
\end{lstlisting}

\paragraph{Lösung}


