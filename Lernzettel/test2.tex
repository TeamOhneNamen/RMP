% !TEX root = rmp_lernzettel.tex

\section{Lösung für Test 2:}
\subsection{}
\begin{lstlisting}
Bytefeld 	DCB 	11,'B', 0xB, 0b01000010
\end{lstlisting}

a)
\begin{lstlisting}
Bytefeld 	0B, 42, 11, 42
\end{lstlisting}
 
b)
\begin{lstlisting}
ldr r1, =Bytefeld
ldrb r0, r1 
\end{lstlisting} 
 
c)
\begin{lstlisting}
0x11, B
\end{lstlisting} 

\subsection{}
\begin{lstlisting}
ALIGN 4
mov r0, 0xAB
\end{lstlisting} 

\subsection{}
\begin{lstlisting}
ldr r0, 0x1256ABCD
\end{lstlisting} 

\subsection{}
\begin{lstlisting}
bgt
\end{lstlisting} 

\subsection{}
\begin{lstlisting}
bgt
\end{lstlisting} 
 
\subsection{}
Aufgabe:\\
Die \textbf{vorzeichenlose} Zahl in r0 soll durch 4 geteilt werden. Das Ergebnis soll in r1 stehen.\\
Geben Sie den Befehl an:\\


\paragraph*{Lösung:}

\begin{lstlisting}
MOV r0,r1,ASR #2
\end{lstlisting}


\paragraph*{Erklärung:}
Eine Verschiebeoperation nach \textbf{links} um 1 Bit entspricht der \textbf{Multiplikation} mit 2 \\
und eine Verschiebeoperation nach \textbf{rechts} um 1 Bit entspricht der \textbf{Division} mit 2. \\
Warum \# 2 statt 4? \# 1 $\rightarrow$ x $\div$ 2; \# 2 $\rightarrow$ x $\div$ 2 $\div$ 2 $\rightarrow$ x $\div$ 4 


\paragraph*{Nachschlagen:}
Kapitel 8.5.5

\subsection{}
Aufgabe:\\
Das Datenfeld Var1 beginne bei Adresse 0x2000. Welcher Wert (hex.) vsteht nach Ausführung des Befehls in r0?\\

\begin{lstlisting}
Var1 	DCB 	10, 'A', 0xA, '1'

		ldr r0, =Var1
\end{lstlisting}

\paragraph*{Lösung}
r0 = 0x2000

\paragraph*{Erklärung:}
Lade die Adresse von Var1 in r0. 


\paragraph*{Nachschlagen:}
Kapitel 7.5.3

\subsection{}
Das Datenfeld Tab beginne bei Adresse 0x2000. Geben Sie die Speicherinhalte (hex.) von Adresse 0x2000 - 0x2003 an?\\
\begin{lstlisting}
Var1 	DCB 	0x10, 'A', 10, '1'
\end{lstlisting}

\paragraph*{Lösung}
0x2000: 41 0A 31 10 \\


\paragraph*{Erklärung:}
0x10 $\rightarrow$ Hexadezimal $\rightarrow$ 10\\
'A' $\rightarrow$ ASCII $\rightarrow$ 41\\
10 $\rightarrow$ Hexadezimal $\rightarrow$ A\\
'1' $\rightarrow$ ASCII $\rightarrow$ 41\\

\paragraph*{Nachschlagen:}\\
Kapitel 7.4.3 Folie 18 $\rightarrow$ Wie werden die Sachen gespeichert?\\
Kapitel 6.4 $\rightarrow$ Reihenfolge im Speicher\\

\subsection{}
Folgendes Datenfeld sei gegeben:
\begin{lstlisting}
Var1 	DCD 	0x10, 0xAA12
\end{lstlisting}
Geben Sie die Assemblerbefehle an, um das \underline{erste Datenwort} des Feldes Var1 nach r1 zu kopieren

\paragraph*{Lösung:}
\begin{lstlisting}
ldr r0, =Var1 ; Arraystartadresse laden 
ldr r1, [r0] ; Erstes Element des Arrays
\end{lstlisting}
\paragraph*{Erklärung:}
Warum nicht mov?\\
mov kopiert nur ein Datenwort Syntax: MOV <wohin>, <woher,was> $\rightarrow$ Daten > \\
Nachschlagen von MOV: Kapitel 6.9.3\\
\paragraph*{Nachschlagen:}
Kapitel 7.5.3
Kapitel 7.7.3


\subsection{}
Was steht in r0 nach folgendem Befehl (hex.)?
\begin{lstlisting}
ldr 		r0, =0x1234ABCD
\end{lstlisting}


\paragraph*{Lösung:}
r0 = 0x1234ABCD

\paragraph*{Erklärung:} 
Wenn nach '=' ein Hexwert kommt dann speichere den Wert. Wenn Variable, dann speichere die Adresse.\\
Auch hier würde mov nicht funktionieren, da 0x1234ABCD > 8 Bit\\


\paragraph*{Nachschlagen:}
http://www.keil.com/support/man/docs/armasm/armasm_dom1361289875065.htm\\
https://www.raspberrypi.org/forums/viewtopic.php?&t=16528\\

\subsection{}
In welchem Wertebereich muss r0 liegen, damit ein Sprung nach LOOP erfolgt? (dezimal oder hex.)
\begin{lstlisting}
mov		r1, #-15
cmp		r0, r1
bge		LOOP	;if greater or equal
\end{lstlisting}

Größer oder gleich:\\
Kleiner oder gleich:\\

\paragraph*{Lösung:}
Größer oder gleich: -15\\
Kleiner oder gleich: 255\\

\paragraph*{Erklärung:}
greater or equal $\rightarrow$ r1 muss >= -15 
mov r1 $\rightarrow$ 8-Bit $\rightarrow$ 1 muss <= 255

\paragraph*{Nachschlagen:}
bge $\rightarrow$ Kapitel 8.3.5

\subsection{}
Was steht in r0 nach folgender Befehlssequenz (hex.)?
\begin{lstlisting}
ldr		r1, =0xFFFFFF87
mov		r0, #0x78
and		r0, r1
\end{lstlisting}

\paragraph*{Lösung:}

\begin{lstlisting}
r1 1111 1111 1111 1111 1111 1111 1000 0111
r0 0000 0000 0000 0000 0000 0000 0111 1000
------------------------------------------
r0 0000 0000 0000 0000 0000 0000 0000 0000
\end{lstlisting}

\paragraph*{Erklärung:}
logisches UND nur wenn gleiche Werte in r1 und r0 stehen $\rightarrow$ 1

\paragraph*{Nachschlagen:}
Kapitel 8.4.3